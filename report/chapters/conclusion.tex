% \clearpage
\section{Conclusion}

	Throughout this project, several aspect of computational quantum physics have been learned. Firstly, the full implementation of DMRG enabled to understand deeper the idea underlying the MPS formalism as well as the ground state search using DMRG. Secondly, the extraction of excitation energies in a very efficient way for critical system. Moreover, controlling the convergence to the ground state via the application of some parity operator or the pinning of the edge spins. Finally, the construction of the MPO for Hamiltonian and how to use the DMRG with periodic boundary conditions in an computationally inefficient way but that does not require any change in the code except the calculation of the corresponding MPO Hamiltonian.

	The main achievements done have been to benchmark the code with TFI model and recover a lot of things to characterize its phases and its critical point described by the Ising CFT. Gaps were computed, spins measured, central charge recovered with several open boundary conditions and periodic ones, conformal towers built and ratios computed. Results obtained for it are very neat and computational time relatively small. Moreover, the idea of a phase diagram for the OF model for values $\lambda_3/\lambda_I \in [0, 1]$ has been found. This was first done by trying to confirm the location of the TCI CFT point at $\lambda_3/\lambda_I=0.856$. However, OBCs do not work to find the central charge expected for the model, even fixing the edge spins. Nonetheless, the PBCs resolved this problem and the correct central charge has been found very precisely. To confirm this, the ratios have been computed and results speak neatly, again via PBCs.

	What was left to do is to introduce the chiral part $\mc H_c$ of the OF model and complete the phase diagram \cite{obrien2018} obtained, since one of the main interest of this model is that is exhibits supersymmetry on the lattice for critical lines $\lambda_c = \pm \lambda_I$. Unfortunately, struggles in finding the correct central charge of the TCI point and small oversight in the PBCs MPO that costed weeks have prevented the study of the addition the chiral term. Moreover, the interest arisen throughout the weeks but the impossibility to further study the theoretical aspects that were involved due the small although substantial amount of time allocated for this project left some kind of bitter taste of incomplete study. 